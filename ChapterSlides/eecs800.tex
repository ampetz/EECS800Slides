\documentclass{beamer}

% theme definition
\usetheme{KU}

\usepackage{natbib}
\usepackage{alltt}
%\usepackage{trust}
\usepackage{fixltx2e}
\usepackage{amsmath}
\usepackage{trust-spi}
\usepackage{tikz}
\usetikzlibrary{arrows,shadows, matrix}
\usepackage[underline=false]{pgf-umlsd}
\usepackage{graphicx}

\usepackage{listings}
\lstset{language=Haskell, emph={main, Request, CAResponse, CACertificate, EvidencePackage, Response, Att,TPM_KEY ,TPM_KEY_USAGE
    , TPM_KEY_FLAGS
    , TPM_AUTH_DATA_USAGE
    , TPM_KEY_PARMS
    , ByteString
    , TPM_STORE_PUBKEY, OIAP, OSAP, TPM_NONCE, TPM_DIGEST, TPM_QUOTE_INFO, TPM_COMPOSITE_HASH, Binary, Get, TPM_STRUCT_VER, TPM, TPM_KEY_HANDLE, TPM_KEY, Session}, emphstyle=\textbf, deletekeywords={putStrLn, Handle}, showstringspaces=false}
\lstset{%
  literate={->}{{ ${\rightarrow}$ }}3 {<-}{{ $\leftarrow$ }}3 
}

\setbeamertemplate{blocks}[rounded][shadow=true]

\setbeamercolor{title}{fg=kublue}
\setbeamercolor{subtitle}{fg=kugray} 
\setbeamercolor{institute}{fg=kugray}
\setbeamercolor{frametitle}{fg=kublue}
\setbeamercolor{frametitle}{bg=white}
\setbeamercolor{framesubtitle}{fg=kugray}
\setbeamercolor{framesubtitle}{bg=white}
\setbeamercolor{item}{fg=black}
\setbeamercolor{subitem}{fg=kugray}
\setbeamercolor{itemize/enumerate subbody}{fg=kugray}
\setbeamercolor{block title}{bg=kublue}
\setbeamercolor{block title}{fg=white}
\setbeamercolor{block body}{bg=sand}
\setbeamercolor{block body}{fg=black}

\setbeamertemplate{footline}[frame number]

\usefonttheme{serif}

\newenvironment{fnverbatim}{\begin{alltt}\scriptsize}{\normalsize\end{alltt}}
\newcommand{\mean}[1]{\langle#1\rangle}
\newcommand{\rtime}{\ensuremath{\mathbb{R}^{0\leq}}}


%constants
\def \app {App}
\def \att {Att}
\def \ca {CA}
\def \mea {Meas}
\def \tp {TPM}
\def \pmask {PCR\textsubscript{m}}
\def \pcomp {PCR\textsubscript{c}}
\def \evd {d}
\def \eve {e}
\def \cacert {\sign {\public{AIK}}{CA}}
\def \exdata {\hash{(\eve, \nonce{\app}, \cacert ) }}
\def \aikh {AIK_h}

\def \req {R}
\def \resp {P}
\def \k {K}

\bibliographystyle{apalike}

\logo{}

\title{ArmoredSoftware: Trust in the cloud}
\subtitle{Annual Demonstration}

\author{Dr. Perry Alexander\inst{1} \and Dr. Andrew Gill\inst{1} \and Dr. Prasad
  Kulkarni\inst{1} \and
  Adam Petz\inst{1} \and Paul Kline\inst{1} \and Justin Dawson\inst{1}
  \and Jason~Gevargizian\inst{1} 
  \and Leon Searl\inst{1} \and Edward Komp\inst{1} \and
  Edward~Bishop\inst{2} \and Ciro Pinto-Coelho\inst{2}} 

\date{{\color{kugray}January 15, 2015}}

% turns off navigation symbols
\setbeamertemplate{navigation symbols}{}

\institute{
  \inst{1}
    Information and Telecommunication Technology Center \\
    Electrical Engineering and Computer Science \\
    The University of Kansas \\
    \medskip
  \inst{2} Southern Cross Engineering}

\begin{document}

\begin{frame}
 \frametitle {Chapter 2, Adam P. and Josh F.}
\end{frame}

\logo{\pgfuseimage{jhwk4C_RF}}


\begin{frame}
 \frametitle{Theories}
  \begin{itemize}
  \item What is a Theory?
    \begin{itemize}
    \item Uses First Order Logic(Variables, Logical symbols, Nonlogical symbols, Syntax) as a generic syntactic framework
    \item Each theory has its own restriction on Nonlogical symbols
    \item INTERPRETATION of Nonlogical symbols is also important
    \item Most of the theories we will see in this book are ``quantifier-free",  and the logical axioms(restrictions on the interpretation of logical symbols) are built-in:  common to all FO theories
    \end{itemize}
   \end{itemize}
\end{frame}


\begin{frame}

 \frametitle{Theory 1:  Propositional Logic}
 
 \begin{itemize}
 
 \item Simple syntax:  Or, Not, (), atom
 \item Three atoms:  Boolean-identifiers, TRUE, FALSE
 \item Many applications:  database queries, planning problems in AI, automated reasoning, circuit design, etc.
 \end{itemize}
 
\end{frame}

\begin{frame}

 \frametitle{Motivation}
 
 \begin{itemize}
 
 \item \ensuremath{n} radio stations
 \item \ensuremath{k} transmission frequencies, \ensuremath{k} \ensuremath{<} \ensuremath{n}
 \item Two stations that are too close together cannot have the same frequency(call this set \ensuremath{E})
 
 \end{itemize}
 
\end{frame}

\begin{frame}

 \frametitle{Motivation}
 
 \begin{itemize}
 
 \item Every station is assigned at least one frequency  
 \item Every station is assigned not more than one frequency 
 \item Close stations are not assigned the same frequency  

 
 \end{itemize}
 
\end{frame}

\begin{frame}

  \frametitle{Motivation - Discussion }

  \begin{itemize}
    \item \textbf{Example} Consider three persons, $A$, $B$, and $C$ who
      must be seated in a row, with the following constraints: 
      $A$ won't sit next to $C$, $A$ won't sit it the left chair
      and $B$ won't sit to the right of $C$.

      Write a propositional formula that is satisfiable $iff$ there
      is a seat assignment which satisfies all constraints.


    \item \textbf{Example 2} Given the following programs, show
      they are equivalent: 

     $!(a || b) ? h :!(a==b) ? f : g$\\
     $!(!a || !b) ? g : (!a \&\& !b) ? h : f$\\

  \end{itemize}


\end{frame}


\begin{frame}

 \frametitle{SAT Solvers}
 
 \begin{itemize}
 
 \item Given a Boolean formula \ensuremath{\beta}, a SAT solver decides whether \ensuremath{\beta} is satisfiable.  If so, reports satisfying assignment
 
 \item REMEMBER:  For this chapter, all inputs are in CNF
 \item Vast improvement of SAT solvers in recent years:  learn from wrong assignments, prune large search spaces quickly, and focusing on "important" variables first
 \end{itemize}
 
\end{frame}

\begin{frame}

 \frametitle{SAT Solvers}
 
 \begin{itemize}
 
  \item  DPLL framework
 
    \begin{itemize}
 
    \item  Traversing and backtracking on a binary tree
    \item Internal nodes represent partial assignments, leaves represent full assignments
    \item Complete(terminates AND returns ``Valid'' when input formula is valid)
 
   \end{itemize}
   
  \item Stochastic search framework
  
    \begin{itemize}
 
    \item  Solver guesses a full assignment
    \item If the formula is evaluated to FALSE under this assignment, starts to flip values of variables according to some (greedy) heuristic
    \item Most are incomplete
 
   \end{itemize}
 
 \end{itemize}
 
\end{frame}

\begin{frame}

 \frametitle{DPLL-Definitions}
 
 \begin{itemize}
 
  \item  \textbf{decision-} assign a value to a variable
 \item  \textbf{decision level-} depth in the binary decision tree in which a decision is made, starting from 1.
  \item  \textbf{ground level-} decision level 0;  clauses with a single literal
   \item  \textbf{satisfied clause-} one or more of its literals are satisfied
   \item  \textbf{conflicting clause-} all of its literals are assigned but not satisfied
   \item  \textbf{unit clause-} not satisfied and all but one of its literals are assigned
   \item  \textbf{unresolved clause-} otherwise
   
 
 
 \end{itemize}
 
\end{frame}

\begin{frame}

 \frametitle{DPLL-Definitions}
 
 \begin{itemize}
 
   \item  \textbf{unit clause rule-} Given a partial assignment under which a clause becomes unit, it \textbf{must} be extended so that it satisfies the unassigned literal
   
  \item For a given unit clause \ensuremath{C} with an unassigned literal \ensuremath{l}, we say that ``\ensuremath{C} implies \ensuremath{l}'' and that \ensuremath{C} is the antecedent clause of l, denoted by \emph{Antecedent}(\ensuremath{l}).
 
 \item If more than one unit clause implies \ensuremath{l}, we refer to the clause that the \textbf{SAT solver used} in order to imply \ensuremath{l} as its antecedent

 
 \item High level diagrams on pg. 30
 
 \end{itemize}
 
\end{frame}

\begin{frame}

 \frametitle{BCP}
 
 \begin{itemize}
 
 \item  Repeated application of \textbf{unit clause rule} until either a conflict or no more implications possible
 \item Best visualized(and modeled) with an \textbf{implication graph}
 \item A \textbf{partial implication graph} is a subgraph which illustrates the BCP at a specific decision level
 
 \end{itemize}
 
\end{frame}

\begin{frame}

 \frametitle{BCP}
 
 \begin{itemize}
 
 \item  After reaching a \textbf{conflict node} k, the ANALYZE-CONFLICT function chooses a ``smart'' \textbf{conflict clause} to add to the list of formula constraints
 
 \item Most competitive solvers design ANALYZE-CONFLICT to generate \textbf{asserting clauses} only.
 \item ANALYZE-CONFLICT also choses the decision level to backtrack to.  According to \textbf{conflict-driven backtracking} strategy, choose the \textbf{second most recent decision level in the conflict clause}
 \end{itemize}
 
\end{frame}

\begin{frame}

 \frametitle{BCP}
 
 \begin{itemize}
 
 \item  Is this process guaranteed to terminate?
 \item  Yes.  It is never the case that the solver enters decision level DL again with the same partial assignment (See node x1 in figures on pg. 33)
 
 \end{itemize}
 
\end{frame}


\begin{frame}
 \frametitle {Chapter 3, Adam P. and Paul K.}
\end{frame}

\begin{frame}

 \frametitle{Theory 2:  Equality Logic(and Uninterpreted Functions)}
 
 \begin{itemize}
 \item 
 \item 
 \item 
 \end{itemize}
 
\end{frame}

\nocite{Coker::Principles-of-R,Haldar:04:Semantic-Remote,Fabrega:1999aa,Ryan:09:Introduction-to}

\begin{frame}
  \frametitle{References}
  \bibliography{demo14}
  D. Kroening and O. Strichman, Decision Procedures: An Algorithmic Point of View, Springer, 2008. ISBN: 978-3-540-74104-6
\end{frame}

\end{document}

